\begin{thebibliography}{99}

\bibitem{oasis-cap-1.2}
OASIS, ``Common Alerting Protocol (CAP) Version 1.2,'' OASIS Standard, Jul.\ 1, 2010. Available: \url{https://docs.oasis-open.org/emergency/cap/v1.2/CAP-v1.2-os.html}

\bibitem{fcc-2018-geo}
Federal Communications Commission, ``Wireless Emergency Alerts; Emergency Alert System,'' \emph{Federal Register}, 83(40), Feb.\ 28, 2018. (Defines matching the target area with no more than 0.1\,mile overshoot.) Available: \url{https://www.federalregister.gov/documents/2018/02/28/2018-03990/wireless-emergency-alerts-emergency-alert-system}

\bibitem{fcc-wea-2023-doc}
Federal Communications Commission, ``Wireless Emergency Alerts; Emergency Alert System,'' \emph{Federal Register}, 88(118), pp.\ 40166--40188, Jun.\ 21, 2023. (Codifies 0.1\,mile accuracy, transmission speed requirements.) Available: \url{https://www.federalregister.gov/documents/2023/06/21/2023-12725/wireless-emergency-alerts-emergency-alert-system}

\bibitem{rand-wea-2023-test}
A.\ M.\ Parker \emph{et al.}, ``Assessing Public Reach of the 2023 National Test of the Wireless Emergency Alerts (WEA) System: Results of a National Survey,'' RAND/HSOAC Research Report RR-A2451-1, 2024. Available: \url{https://www.rand.org/pubs/research_reports/RRA2451-1.html}

\bibitem{mcbride-2023-wea-latency}
S.\ K.\ McBride, R.\ Allen, S.\ Baltay, \emph{et al.}, ``Latency and geofence testing of Wireless Emergency Alerts for ShakeAlert,'' \emph{Safety Science}, vol.\ 157, 2023, Art.\ 105999. doi:\,10.1016/j.ssci.2022.105999

\bibitem{kleppmann-2019-localfirst}
M.\ Kleppmann, A.\ Wiggins, N.\ Zeldovich, ``Local-First Software: You Own Your Data, in spite of the Cloud,'' \emph{Onward! 2019}, pp.\ 154--178. doi:\,10.1145/3359591.3359737

\bibitem{mec-caching-survey-2023}
T.\ V.\ Nguyen, A.\ T.\ Tran, N.\ N.\ Dao, H.\ Moon, S.\ Cho, ``Information fusion on delivery: A survey on the roles of mobile edge caching systems,'' \emph{Information Fusion}, vol.\ 89, pp.\ 486--509, Jan.\ 2023. doi:\,10.1016/j.inffus.2022.08.029

\bibitem{icn-iot-caching-survey-2023}
C.\ N.\ Pruthvi, H.\ S.\ Vimala, J.\ Shreyas, ``A systematic survey on content caching in ICN and ICN-IoT: Challenges, approaches and strategies,'' \emph{Computer Networks}, vol.\ 233, 2023, Art.\ 109896. doi:\,10.1016/j.comnet.2023.109896

\bibitem{cache-mab-2023}
S.\ M.\ A.\ Iqbal, M.\ Asaduzzaman, ``Cache-MAB: A reinforcement learning–based hybrid caching scheme in Named Data Networks,'' \emph{Future Generation Computer Systems}, vol.\ 147, pp.\ 163--178, 2023. doi:\,10.1016/j.future.2023.04.032

\bibitem{electronics-2024-koide-icanet}
M.\ Koide, Y.\ Ohira, Y.\ Hirota, ``Caching for Information-Centric Ad Hoc Networks using Popularity and Node Centrality,'' \emph{Electronics}, vol.\ 13, no.\ 11, 2024, Art.\ 2213. doi:\,10.3390/electronics13112213

\bibitem{hotnets-2024-freshness}
Z.\ Mao, R.\ Iyer, S.\ Shenker, I.\ Stoica, ``Revisiting Cache Freshness for Emerging Real-Time Applications,'' in \emph{Proc.\ ACM HotNets 2024}. doi:\,10.1145/3696348.3696858

\bibitem{einziger-2017-wtinylfu}
G. Einziger, R. Friedman, ``TinyLFU: A Highly Efficient Cache Admission Policy,'' \emph{IEEE Trans. on Knowledge and Data Engineering}, vol. 29, no. 4, pp. 826--841, 2017. doi:\,10.1109/TKDE.2016.2632726

\bibitem{shevchenko-2023-geofencing}
V.\ Shevchenko, M.\ Rabinovich, A.\ Shoval, \emph{et al.}, ``Geofencing in location-based behavioral research,'' \emph{Behavior Research Methods}, 2024. doi:\,10.3758/s13428-024-02440-3

\bibitem{mileti-1990-ornl6609}
D.\ S.\ Mileti, J.\ H.\ Sorensen, ``Communication of Emergency Public Warnings: A Social Science Perspective and State-of-the-Art Assessment,'' Oak Ridge National Laboratory Report ORNL-6609, Aug.\ 1990. doi:\,10.2172/6137387

\bibitem{cohen-1985-socialsupport}
S.\ Cohen, T.\ A.\ Wills, ``Stress, social support, and the buffering hypothesis,'' \emph{Psychological Bulletin}, vol.\ 98, no.\ 2, pp.\ 310--357, 1985.

\bibitem{norris-2008-resilience}
F.\ H.\ Norris, S.\ P.\ Stevens, B.\ Pfefferbaum, K.\ F.\ Wyche, R.\ L.\ Pfefferbaum, ``Community Resilience as a Metaphor, Theory, Set of Capacities, and Strategy for Disaster Readiness,'' \emph{American Journal of Community Psychology}, vol.\ 41, pp.\ 127--150, 2008. doi:\,10.1007/s10464-007-9156-6

\bibitem{paton-2008-warningresponse}
D.\ Paton, ``Risk communication and natural hazard mitigation: How trust influences its effectiveness,'' \emph{International Journal of Global Environmental Issues}, vol.\ 8, nos.\ 1--2, pp.\ 2--16, 2008. doi:\,10.1504/IJGENVI.2008.017256

\bibitem{nasem-2018-alerts}
National Academies of Sciences, Engineering, and Medicine, ``Emergency Alert and Warning Systems: Current Knowledge and Future Research Needs,'' Washington, DC: The National Academies Press, 2018. doi:\,10.17226/24935

\bibitem{lindell-2012-padm}
M.\ K.\ Lindell, R.\ W.\ Perry, ``The Protective Action Decision Model: Theoretical Modifications and Additional Evidence,'' \emph{Risk Analysis}, vol.\ 32, no.\ 4, pp.\ 616--632, 2012. doi:\,10.1111/j.1539-6924.2011.01647.x

\end{thebibliography}
